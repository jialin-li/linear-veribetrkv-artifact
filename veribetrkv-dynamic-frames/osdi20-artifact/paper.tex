\documentclass[10pt,twocolumn]{article}

\usepackage{xspace}      % Intelligently adds space after a word via \xspace
\usepackage{subcaption}  % Place two related figures side-by-side
\usepackage{tikz}
\usepackage{pgfplots}
\usetikzlibrary{patterns}

\newcommand{\name}{VeriSafeKV\xspace} % blinded for submission
\newcommand{\namelinear}{VeriSafeKV-Linear\xspace} % blinded for submission

\newcommand{\btree}{B-tree\xspace}
\newcommand{\btrees}{B-trees\xspace}
\newcommand{\Btree}{B-Tree\xspace}
\newcommand{\Btrees}{B-Trees\xspace}
\newcommand{\beptree}{B\ensuremath{^\epsilon}-tree\xspace}
\newcommand{\beptrees}{B\ensuremath{^\epsilon}-trees\xspace}
\newcommand{\Beptree}{B\ensuremath{^\epsilon}-Tree\xspace}
\newcommand{\Beptrees}{B\ensuremath{^\epsilon}-Trees\xspace}
\newcommand{\circnum}[1]{\raisebox{.5pt}{\textcircled{\raisebox{-.9pt} {#1}}}}
\newcommand{\iosystem}{IOSystem\xspace}
\newcommand{\template}[1]{$\langle$#1$\rangle$}

%%%%%%%%%  Paper Info %%%%%%%%%%%%%%%%%%%%%%%%%%%%%%%%%%%%%%%%%%%

\title{\vspace{-2mm}
Storage Systems are Distributed Systems\\
(So Verify Them That Way!)
\\\vspace{2mm}{\large\textbf{OSDI 2020 Artifact Reprodcubility Experiment Results}}
}

\date{}

%%%%%%%%%  Main Body %%%%%%%%%%%%%%%%%%%%%%%%%%%%%%%%%%%%%%%%%%%

\begin{document}
\maketitle

\input{build/linear-line-counts}

\begin{figure}
\begin{center}
\input{build/Impl/Bundle.i.lcreport}
\end{center}
  \vspace{-3mm}
\caption{Line counts by major components.}
\label{line-counts}
\end{figure}

\begin{figure}
  \begin{center}
    \input{build/linear-line-count-table}
  \end{center}
  \vspace{-3mm}
  \caption{Line counts of two subcomponents.
  }
  \label{line-counts-micro}
  \vspace{-2mm}
\end{figure}

\begin{figure}
\includegraphics[width=1\columnwidth]{build/verification-times.pdf}
  \vspace{-5mm}
\caption{Cumulative distribution of verification times of function
  definitions, implementation methods, and proof lemmas.
  }
  \vspace{-4mm}
\label{verification-times}
\end{figure}

\begin{figure*}
\includegraphics[width=1\textwidth,ext=.i.status.pdf,type=pdf,read=*]{build/Impl/Bundle}
\caption{Verification results.}
\label{verification-results}
\end{figure*}

\begin{figure}
\includegraphics[width=1\columnwidth]{build/automation-figure.pdf}
  \vspace{-4mm}
  \caption{Distribution of definition revelations.}
  \vspace{-3mm}
\label{automation-histogram}
\end{figure}

\newcommand{\ycsbHddDataDir}{data/ycsb-hdd/golden}
\newcommand{\ycsbSsdDataDir}{data/ycsb-ssd/golden}
\newcommand{\mutablebtreeDataDir}{data/mutablebtree/silver}

\pgfplotsset{
  GenericBarplot/.style={
    ybar=1pt,
    bar width=5.5pt,
    height = 1.75in,
    enlarge x limits=0.1,
    major x tick style = transparent,
    scaled y ticks=false,
    ylabel near ticks,
    point meta=rawy,
    nodes near coords,
    every node near coord/.append style={rotate=90, anchor=west, font=\tiny},
    nodes near coords={\pgfmathprintnumber[fixed, precision=0]{\pgfplotspointmeta}},
    area legend,
    legend columns=-1,
    label style={font=\small, align=center},
  },
  MinMidMaxFile/.style n args={3}{
    /pgfplots/error bars/y dir=both,
    /pgfplots/error bars/y explicit,
    table/y error plus expr = (\thisrow{#3} - \thisrow{#2}),
    table/y error minus expr = (\thisrow{#2} - \thisrow{#1}),
  }
}

\pgfplotsset{
  YcsbThroughput/.style={
    GenericBarplot,
    symbolic x coords = {Load, A, B, C, D, F, Cuniform},
    ylabel={Operations/Second},
    MinMidMaxFile={MinThroughput}{MedianThroughput}{MaxThroughput},
  },
  YcsbLoadThroughput/.style={
    YcsbThroughput,
    width = 1.1in,
    restrict x to domain=0:0,
    xtick = {Load},
    xlabel={(from YCSB A)},
  },
  YcsbRunThroughput/.style={
    YcsbThroughput,
    width = 3.1in,
    xmin = A,
    xmax = Cuniform,
    restrict x to domain=1:10,
    xtick = {A, B, C, D, F, Cuniform},
    xticklabels = {A, B, C, D, F, U},
    xlabel={YCSB Workload},
  },
}

\pgfplotsset{
    LinearStyle/.style={teal, fill=teal!10!white, postaction={pattern=north west lines, pattern color=teal!50!white}},
    HeapStyle/.style={violet!50!black, fill=violet!50!black!10!white, postaction={pattern=north east lines, pattern color=violet!50!black!50!white}},
}

\pgfplotsset{
    BerkeleyStyle/.style={blue, fill=blue!10!white, postaction={pattern=crosshatch dots, pattern color=blue!50!white}},
    VeriHeapStyle/.style={HeapStyle},
    VeriLinearStyle/.style={LinearStyle},
    RocksStyle/.style={red, fill=red!05!white, postaction={pattern=crosshatch, pattern color=red!50!white}},
}

\begin{figure*}
\centering
\ref{ycsb-legend}
  \begin{tikzpicture}
    \begin{axis}[
        YcsbLoadThroughput,
        ymode = log,
        %ymax = 100000,
        %ytick = { 100, 1000, 10000, 100000, 1000000 },
        legend columns=1,
        legend to name={ycsb-legend},
      ]
      \addplot+ [BerkeleyStyle] table {build/BerkeleyYcsb.csv};
      \addlegendentry{BerkeleyDB}
      \addplot+ [VeriHeapStyle] table {build/VeribetrfsYcsb.csv};
      \addlegendentry{\name-DF}
      \addplot+ [RocksStyle] table {build/RocksYcsb.csv};
      \addlegendentry{RocksDB}
    \end{axis}
  \end{tikzpicture}
  \begin{tikzpicture}
    \begin{axis}[
        YcsbRunThroughput,
        %ymin = 0,
        %ymax = 800,
      ]
      \addplot+ [BerkeleyStyle] table {build/BerkeleyYcsb.csv};
      \addplot+ [VeriHeapStyle] table {build/VeribetrfsYcsb.csv};
      \addplot+ [RocksStyle] table {build/RocksYcsb.csv};
    \end{axis}
  \end{tikzpicture}

\caption{
Median throughput of YCSB workloads values.  Load is 10M operations
($\approx$ 5GB of data) and runs are 10000 operations each.
Higher is better.
}
\label{fig:ycsb}
\end{figure*}

\begin{figure}
\centering
%\ref{micro-legend}
\begin{subfigure}[t]{.25\textwidth}
  \begin{tikzpicture}
    \begin{axis}[
        GenericBarplot,
        enlarge x limits=0.2,
        width = 1.8in,
        symbolic x coords = {insert, readpositive, readnegative, remove},
        xtick = {insert, readpositive, readnegative, remove},
        xticklabels = {Insert, Pos. Query, Neg. Query, Delete},
        x tick label style={rotate=90},
        ymin = 0,
        %ymax = 8300000,
        %ytick = {0, 2000000, 4000000, 6000000, 8000000},
        %yticklabels={0, 2M, 4M, 6M, 8M},
        ylabel={Operations/Second},
        legend to name={micro-legend},
        MinMidMaxFile={min}{med}{max},
      ]
      \addplot+ [HeapStyle] table {build/mutable-map-benchmark.csv};
    \end{axis}
  \end{tikzpicture}
  \subcaption{Hash table}
  \label{subfig:micro-hashtable}
\end{subfigure}~%
\hspace{0.3in}
\begin{subfigure}[t]{.2\textwidth}
  \centering
  \parbox{1.7\textwidth}{
    \vspace{-2.475in}
  \begin{tikzpicture}
    \begin{axis}[
        GenericBarplot,
        enlarge x limits=0.5,
        width = 1.3in,
        symbolic x coords = {write,read},
        xtick = {write, read},
        xticklabels = {Insert, Query},
        x tick label style={rotate=90},
        ymin = 0,
        %ymax = 300000,
        %ylabel={Operations/Second},
        MinMidMaxFile={min}{med}{max},
      ]
      \addplot+ [HeapStyle] table {build/mutable-btree-benchmark.csv};
    \end{axis}
  \end{tikzpicture}}\\
  \subcaption{Search tree}
  \label{subfig:micro-btree}
\end{subfigure}
\caption{Throughput of subcomponent microbenchmarks.
  Higher is better. 
}\label{fig:datastructures-micro}
\end{figure}


\end{document}
